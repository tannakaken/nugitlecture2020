\documentclass[../main]{subfiles}

\begin{document}
 \title{ 2010年名古屋大学入試理系数学(前期日程)大問4}
 \author{}
 \date{}
 \maketitle

  (1)
  $ y = \frac{1}{3}x^{2}+ \frac{1}{2}x$
  のグラフ上に無限個の格子点が存在する 
  ことを示せ.\\
  
  解)mを任意の整数とし$x=6m$とおく.\\
  xを与えられた式に代入すると\\
   \begin{equation*}   
    y &= \frac{1}{3}\times36m^{2}+\frac{1}{2}\times6m
    &= 3m(4m+1)
   \end{equation*}
  よって$(6m,3m(4m+1))$は格子点.\\
  したがって整数mは無限個存在するため,格子点は無限個存在する.\\

  (2)a,bは実数でa\ne0とする.$y=ax^{2}+bx$のグラフ上に点$(0,0)$以外
  に格子点が2つ存在すれば、無限個存在することを示せ.\\

  解)2点(p,q),(r,s)を格子点とする.それぞれ与えられた式に代入すると,\\
   \begin{cases}
    q&=ap^{2}+bp\mbox{…(あ)}\\
    s&=ar^{2}+br\mbox{…(い)}
   \end{cases}

  $(あ)\times r-(い)\times p $より
   \begin{equation*}
    qr-ps=a(p^{2}r-pr^{2})
   \end{equation}
  よって
   \begin{equation*}
    a=\frac{-(ps-qr)}{pr(p-r)}
   \end{equation}
  同様に
  $(あ)\times r^{2}-(い)\times p^{2} $より
   \begin{equation*}
    qr^{2}-p^{2}s=b(r^{2}p-rp^{2})
   \end{equation*}
  よって
   \begin{equation*}
    b=\frac{-(p^{2}s-qr^{2})}{pr(p-r)}
   \end{equation*}
  したがって最初に与えられた式は、
   \begin{equation*}
    y=\frac{-(ps-qr)}{pr(p-r)}x^{2}+\frac{-(p^{2}s-qr^{2})}{pr(p-r)}x
   \end{equation*}
  とかける.
  ここで(1)と同様に、nを任意の整数とし$x=pr(p-r)n$とおく.\\
  xを代入すると,
   \begin{equation*}
    y=-(ps-qr)pr(p-r)n^{2}+(p^{2}s-qr{2})n
   \end{equation*}
  よって$(pr(p-r)n,-(ps-qr)pr(p-r)n^{2}+(p^{2}s-qr{2})n)$は格子点.\\
  したがって整数nは無限個存在するため,格子点は無限個存在する.\\
     
\end{document}
