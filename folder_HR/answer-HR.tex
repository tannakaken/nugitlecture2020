\documentclass[../main]{subfiles}

\begin{document}
 \title{ 2010年名古屋大学入試理系数学(前期日程)大問4}
 \author{}
 \date{}
 \maketitle

  (1)
  $ y = \frac{1}{3}x^{2}+ \frac{1}{2}x$
  のグラフ上に無限個の格子点が存在する 
  ことを示せ.\\
  
  解)mを任意の整数とし$x=6m$とおく.\\
  xを与えられた式に代入すると\\
    $y = \frac{1}{3}\times36m^{2}+\frac{1}{2}\times6m
    = 3m(4m+1)$\\
  よって$(6m,3m(4m+1))$は格子点.\\
  したがって整数mは無限個存在するため,格子点は無限個存在する.\\

  (2)a,bは実数でa\ne0とする.$y=ax^{2}+bx$のグラフ上に点$(0,0)$以外
  に格子点が2つ存在すれば、無限個存在することを示せ.\\

  解)2点(p,q),(r,s)を格子点とする.それぞれ与えられた式に代入すると,\\
   \begin{cases}
    q&=ap^{2}+bp\mbox{…(あ)}\\
    s&=ar^{2}+br\mbox{…(い)}
   \end{cases}

  $(あ)\times r-(い)\times p $より
   $qr-ps=a(p^{2}r-pr^{2})$
   
  よって
   $a=\frac{-(ps-qr)}{pr(p-r)}$\\
   同様に
  $(あ)\times r^{2}-(い)\times p^{2} $より
   $qr^{2}-p^{2}s=b(r^{2}p-p^{2}r)$\\
  よって
   $b=\frac{-(p^{2}s-qr^{2})}{pr(p-r)}$\\
   したがって最初に与えられた式は、
   $y=\frac{-(ps-qr)}{pr(p-r)}x^{2}+\frac{-(p^{2}s-qr^{2})}{pr(p-r)}x$\\
   とかける.\\
  ここで(1)と同様に、nを任意の整数とし$x=pr(p-r)n$とおく.\\
  xを代入すると,   
    $y=-(ps-qr)pr(p-r)n^{2}+(p^{2}s-qr{2})n$\\
  よって$(pr(p-r)n,-(ps-qr)pr(p-r)n^{2}+(p^{2}s-qr{2})n)$は格子点.\\
  したがって整数nは無限個存在するため,格子点は無限個存在する.\\
     
\end{document}
