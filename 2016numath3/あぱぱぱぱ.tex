\documentclass{jsarticle}
\usepackage{enumerate}
\begin{document}
\section*{2016年度 名古屋大学 文系数学 大問3}
正の整数$n$に対して、その($1$と自分自身も含めた)全ての正の約数の和を$s(n)$とかくことにする。このとき、次の問いに答えよ。
\begin{enumerate}[(1)]
\items$ kを正の整数、pを3以上の素数とするとき、s(2^k*p)を求めよ。$
\items $s(2016)を求めよ。$
\items $2016の正の約数nで、s(n)=2016となるものをすべて求めよ。$
\end{enumerate}
\subsection*{解答}
(!)2^kの正の約数は1,2,2^2,...2^k,またpの正の約数は1,pである\\
s(2^k*p)=(1+2+2^2+...+2^k)*(1+p)=(2^(k-1))*(1+p)\\
(2)2016=2^5*3^2*7\\
s(2016)=(1+2+2^2+2^3+2^4+2^5)*(1+3+3^2)*(1+7)=6552\\
(3)s(n)=(2^x-1)*\frac{(3^y-1)}{2}*\frac{(7^z-1)}{6} \\
(1\leq$x$\leq6,1\leq$y$\leq3,1\leq$z$\leq2,x,y,zは整数)...a\\
s(n)=2016=2^5*3^2*7...b\\
a,bより、(2^x-1)*(3^y-1)*(7^z-1)=2^7*3^3*7...c\\
()z=1の時\\
cより(2^x-1)*(3^y-1)=2^6*3^2*7\\
2^x-1は奇数かつ3^y-1は偶数より\\
3^y-1=2^6=64\\
これを満たす正の整数yは存在しない。\\
()z=2の時\\
cより(2^x-1)*(3^y-1)=2^3*3^2*7
(あ)と同様にして\\
2^x-1=3^2*7\\
3^y-1=2^3\\
x=6,y=2はこれを満たす。\\
以上より、求めるnは2^5*3*7=672\\
\end{document}
