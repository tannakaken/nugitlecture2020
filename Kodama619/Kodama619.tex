\documentclass[../main]{subfiles}


\begin{document}
\begin{center}
	{\large 2020名古屋大学入試理系数学大問4}
\end{center}
\begin{flushright}
  322001115~~児玉悠太
\end{flushright}
\subsection*{問題}
2名が先攻と後攻にわかれ,次のようなゲームを行う.
\begin{enumerate}[(i)]
	\item 正方形の4つの頂点を反時計回りにA,B,C,Dとする.
				両者はコマを1つずつ持ち,ゲーム開始時には先攻の持ちゴマはA,後攻の持ちゴマはCにおいてあるとする.
	\item 先攻から始めて,交互にサイコロを振る.
				ただしサイコロは1から6までの目が等確率で出るものとする.
				出た目を3で割ったあまりが0のときコマは動かさない.
				また,余りが1のときは,自分のコマを反時計回りに隣りの頂点に動かし,余りが2のときは,自分のコマを時計回りに隣りの頂点に動かす.
				もし移動した先に相手のコマがあれば,その時点でゲームは終了とし,サイコロを振った者の勝ちとする.
\end{enumerate}
ちょうど$n$回サイコロが振られたときに勝敗が決まる確率を$p_n$とする.
このとき,以下の問いに答えよ.
\begin{enumerate}[(1)]
	\item $p_2,p_3$を求めよ.
	\item $p_n$を求めよ.
	\item このゲームは後攻にとって有利であること,すなわち2以上の任意の整数$N$に対して,$\displaystyle \sum_{m=1}^{[\frac{N+1}{2}]}p_{2m-1}<\sum_{m=1}^{[\frac{N}{2}]}p_{2m}$が成り立つことを示せ.
				ただし正の実数$a$に対し$[a]$はその整数部分($k\leqq a<k+1$となる整数$k$)を表す.
\end{enumerate}

\subsection*{解答}
\begin{enumerate}[(1)]
\item ルール(ii)よりサイコロを振って,3,6が出たときコマは移動せず,1,4が出たときコマは反時計回り隣りの頂点に移動し,2,5が出たときコマは時計回りに隣りの頂点に移動する.
従って,サイコロを振ってコマが「移動しない」,「反時計回り隣りの頂点に移動する」,「時計回り隣りの頂点に移動する」確率はすべて1/3ずつである.
				
			ここで2回サイコロが振られて勝敗が決まるのは,「1回目で先攻のコマがAからBへ移動し,2回目で後攻のコマがCからBへ移動する」か「1回目で先攻のコマがAからDへ移動し,2回目で後攻のコマがCからDへ移動する」かのどちらかであるから,サイコロを2回振って勝敗が決まる確率$p_2$は,
			\[
				p_2=2\cdot \left(\frac{1}{3}\right)^2=\frac{2}{9}
			\]
				
			一方3回サイコロが振られて勝敗が決まるのは,「1回目で先攻のコマがAから動かず,2回目で後攻のコマがCからBへ移動し,3回目で先攻のコマがAからBへ移動する」か「1回目で先攻のコマがAから動かず,2回目で後攻のコマがCからDへ移動し,3回目で先攻のコマがAからDへ移動する」か「1回目で先攻のコマがAからBへ移動し,2回目で後攻のコマがCから動かず,3回目で先攻のコマがCからBへ移動する」か「1回目で先攻のコマがAからDへ移動し,2回目で後攻のコマがCから動かず,3回目で先攻のコマがCからDへ移動する」かの4つのうちのどれかであるから,サイコロを3回振って勝敗が決まる確率$p_3$は,
			\[
				p_3=4\cdot \left(\frac{1}{3}\right)^3=\frac{4}{27}
			\]

\item ちょうど$n$回サイコロが振られたときにコマが隣り合う確率を$q_n$,コマが隣り合わない且つ勝敗も決まらない確率を$r_n$とする.
			このとき,$p_0=r_0=0,q_0=1$と漸化式
			\begin{numcases}
  			{}
 				\label{p=r/3}
				p_{n+1}=\frac{1}{3}r_n & \\
				\label{q=q/3+r/3}
				q_{n+1}=\frac{1}{3}q_n+\frac{1}{3}r_n & \\
				\label{r=2q/3+r/3}
				r_{n+1}=\frac{2}{3}q_n+\frac{1}{3}r_n &
			\end{numcases}
			が成り立つ.
			
			式(\ref{r=2q/3+r/3})より,$q_n=(3r_{n+1}-r_n)/2$であり,これを式(\ref{q=q/3+r/3})に代入,整理すると,
			\begin{equation}
				\label{r=2r/3+r/9}
				r_{n+2}=\frac{2}{3}r_{n+1}+\frac{1}{9}r_n
			\end{equation}
			ここで式(\ref{r=2r/3+r/9})の特性方程式$t^2=\cfrac{2}{3}t+\cfrac{1}{9}$の解$t=\cfrac{1\pm \sqrt{2}}{3}$について,$\alpha=\cfrac{1-\sqrt{2}}{3},\beta=\cfrac{1+\sqrt{2}}{3}$とおくと,式(\ref{r=2r/3+r/9})は,
			\[
				r_{n+2}-\alpha r_{n+1}=\beta (r_{n+1}-\alpha r_n),
				r_{n+2}-\beta r_{n+1}=\alpha (r_{n+1}-\beta r_n)
			\]
			と2つの形に書き換えられる.
			式(\ref{p=r/3})より,$r_1=3p_2=2/3$であるから,数列$\{r_{n+1}-\alpha r_n\}_{n=0}^{\infty}$は初項$r_1-\alpha r_0=2/3$,公比$\beta$の等比数列,数列$\{r_{n+1}-\beta r_n\}_{n=0}^{\infty}$は初項$r_1-\beta r_0=2/3$,公比$\alpha$の等比数列であるから,
			\[
				r_{n+1}-\alpha r_n=\frac{2}{3}\beta^n,
				r_{n+1}-\beta r_n=\frac{2}{3}\alpha^n
			\]
			辺々引いて,$(-\alpha+\beta)r_n=2(\beta^n-\alpha^n)/3$であり,$-\alpha+\beta=-(1-\sqrt{2})/3+(1+\sqrt{2})/3=2\sqrt{2}/3$であるから,
			\[
				r_n=\frac{3}{2\sqrt{2}}\frac{2(\beta^n-\alpha^n)}{3}=\frac{\sqrt{2}}{2}(\beta^n-\alpha^n)
			\]
			よって式(\ref{p=r/3})から$n\geqq1$において
			\[
				p_n=\frac{1}{3}r_{n-1}
				=\frac{\sqrt{2}}{6}(\beta^{n-1}-\alpha^{n-1})
				=\frac{\sqrt{2}}{6}\left(\left(\frac{1+\sqrt{2}}{3}\right)^{n-1}-\left(\frac{1-\sqrt{2}}{3}\right)^{n-1}\right)
			\]
			以上より,求める解は
			\begin{equation}
				\label{p=0,sqrt2/6etc.}
			  p_n = \begin{cases}
			    0 & (n=0) \\
			    \cfrac{\sqrt{2}}{6}\left(\left(\cfrac{1+\sqrt{2}}{3}\right)^{n-1}-\left(\cfrac{1-\sqrt{2}}{3}\right)^{n-1}\right) & (n=1,2,\dots)
			  \end{cases}
			\end{equation}
\item 任意の自然数$n$に対し,(\ref{p=0,sqrt2/6etc.})より,
			\begin{eqnarray*}
				p_{2n}-p_{2n+1}&=&\frac{\sqrt{2}}{6}(\beta^{2n-1}-\alpha^{2n-1})-\frac{\sqrt{2}}{6}(\beta^{2n}-\alpha^{2n})\\
				&=&\frac{\sqrt{2}}{6}(\beta^{2n-1}(1-\beta)-\alpha^{2n-1}(1-\alpha))\\
				&=&\frac{\sqrt{2}}{6}\left(\frac{2-\sqrt{2}}{3}\beta^{2n-1}-\frac{2+\sqrt{2}}{3}\alpha^{2n-1}\right)
			\end{eqnarray*}
			ここで$\alpha<0<\beta$より,$(2-\sqrt{2})\beta^{2n-1}/3,-(2+\sqrt{2})\alpha^{2n-1}/3>0$であるから,$p_{2n}-p_{2n+1}>0$,すなわち$p_{2n+1}<p_{2n}$が成り立つ.
			
			ここで,$k$を2以上の自然数とし,$N$について(i)$N=2$のとき,(ii)$N=2k$のとき,(iii)$N=2k-1$のとき,の3つに場合分けし$\displaystyle \sum_{m=1}^{[\frac{N+1}{2}]}p_{2m-1}<\sum_{m=1}^{[\frac{N}{2}]}p_{2m}\cdots (\ast)$が成り立つことを示す.
			\begin{enumerate}[(i)]
				\item $N=2$のとき
							
							$0=p_1<p_2=2/9$より,$(\ast)$が成り立つことは明らか.
				\item $N=2k$のとき
				
							$p_1=0$より,$\displaystyle \sum_{m=1}^{[\frac{N+1}{2}]}p_{2m-1}=\sum_{m=1}^kp_{2m-1}=\sum_{m=2}^kp_{2m-1}$.
							また,$\sum_{m=1}^{[\frac{N}{2}]}p_{2m}=\sum_{m=1}^kp_{2m}$であるから,
							\[
								\sum_{m=1}^{[\frac{N+1}{2}]}p_{2m-1}
								=\sum_{m=2}^kp_{2m-1}
								<\sum_{m=1}^{k-1}p_{2m}
								<\sum_{m=1}^kp_{2m}
								=\sum_{m=1}^{[\frac{N}{2}]}p_{2m}
							\]
							よって$(\ast)$は成り立つ.
				\item $N=2k-1$のとき
							
							$p_1=0$より,$\displaystyle \sum_{m=1}^{[\frac{N+1}{2}]}p_{2m-1}=\sum_{m=1}^{k+1}p_{2m-1}=\sum_{m=2}^{k+1}p_{2m-1}$.
							また,$\sum_{m=1}^{[\frac{N}{2}]}p_{2m}=\sum_{m=1}^kp_{2m}$であるから,
							\[
								\sum_{m=1}^{[\frac{N+1}{2}]}p_{2m-1}
								=\sum_{m=2}^{k+1}p_{2m-1}
								<\sum_{m=1}^kp_{2m}
								=\sum_{m=1}^{[\frac{N}{2}]}p_{2m}
							\]
							よって$(\ast)$は成り立つ.
			\end{enumerate}
			以上(i)(ii)(iii)より,2以上の任意の整数$N$に対して,$\displaystyle \sum_{m=1}^{[\frac{N+1}{2}]}p_{2m-1}<\sum_{m=1}^{[\frac{N}{2}]}p_{2m}$が成り立つことが示された.
			
\end{enumerate}


\end{document}