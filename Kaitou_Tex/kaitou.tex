\documentclass[11pt,a4paper]{jsarticle}
\usepackage{amsmath,amssymb}
\usepackage{bm}
\usepackage{graphicx}
\usepackage{ascmac}
\usepackage{enumerate}
\begin{document}
\noindent
{\bf 2018年 名古屋大学(理系)前期日程 大問1}

\vspace{5mm}
自然数$n$に対し,定積分$\displaystyle I_{n}=\int_{0}^{1} \frac{x^{n}}{x^{2}+1} d x$を考える。
このとき,次の問いに答えよ。
\begin{enumerate}[(1)]
 \item $\displaystyle I_{n}+I_{n+2}=\frac{1}{n+1}$ を示せ。
 \item $\displaystyle 0 \leqq I_{n+1} \leqq I_{n} \leqq \frac{1}{n+1}$ を示せ。
 \vspace{2mm}
 \item $\displaystyle \lim _{n \rightarrow \infty} n I_{n}$ を求めよ。
 \item $\displaystyle S_{n}=\sum_{k=1}^{n} \frac{(-1)^{k-1}}{2 k}$とする。
 このとき,(1),(2)を用いて$\displaystyle \lim _{n \rightarrow \infty} S_{n}$を求めよ。
\end{enumerate}

\vspace{20mm}
\noindent
{\bf (1)の解答}

$\displaystyle I_{n}=\int_{0}^{1} \frac{x^{n}}{x^{2}+1} d x$より,$\displaystyle I_{n+2}=\int_{0}^{1} \frac{x^{n+2}}{x^{2}+1} d x$である。和をとると次のようになる。
\[\begin{aligned}
  I_{n}+I_{n+2} &=\int_{0}^{1} \frac{x^{n}}{x^{2}+1} d x+\int_{0}^{1} \frac{x^{n+2}}{x^{2}+1} d x \\
  &=\int_{0}^{1} \frac{x^{n}\left(x^{2}+1\right)}{x^{2}+1} d x \\
  &=\int_{0}^{1} x^{n} d x.
\end{aligned}\]
最後の積分を計算すると次のようになる。
\[\int_{0}^{1} x^{n} d x=\left[\frac{1}{n+1} x^{n+1}\right]_{0}^{1}=\frac{1}{n+1}.\]
よって
\[I_{n}+I_{n+2}=\frac{1}{n+1}\]
が成り立つ。
\\
\\
{\bf (2)の解答}

$\displaystyle I_{n}=\int_{0}^{1} \frac{x^{n}}{x^{2}+1} d x$より,$\displaystyle I_{n+1}=\int_{0}^{1} \frac{x^{n+1}}{x^{2}+1} d x$である。$0\leq x\leq 1$において,$I_n$と$I_{n+1}$の被積分関数を比較すると
\[\frac{x^{n+1}}{x^{2}+1} \leq \frac{x^{n}}{x^{2}+1}\]
が成り立つ。また$0\leq x\leq 1$で$x^2+1>0$かつ$x^{n+1}\geq 0$であるから
\[0 \leq \frac{x^{n+1}}{x^{2}+1} \leq \frac{x^{n}}{x^{2}+1}\]
である。$x$について0から1の範囲で積分することにより
\[0 \leq \int_{0}^{1} \frac{x^{n+1}}{x^{2}+1} d x \leq \int_{0}^{1} \frac{x^{n}}{x^{2}+1} d x\]
が成り立つ。つまり,$0\leq I_{n+1}\leq I_{n}$である。

次に$0\leq I_{n+1}\leq I_{n}$から,$0\leq I_{n+2}\leq I_{n+1}$であることがわかる。つまり$I_{n+2}\geq 0$かつ$I_{n}\geq 0$である。そして (1)より$I_{n}+I_{n+2}=1/(n+1)$であるから,左辺の2項が0以上であることを考慮すれば
\[I_{n} \leq \frac{1}{n+1}\]
である。以上より
\[0 \leq I_{n+1} \leq I_{n} \leq \frac{1}{n+1}\]
が成り立つ。
\\
\\
{\bf (3)の解答}

(2)より$0 \leq I_{n+1} \leq I_{n} \leq 1/(n+1)$が成り立つ。これより
\[0 \leq I_{n+2} \leq I_{n+1} \leq I_{n} \leq I_{n-1} \leq I_{n-2} \ (ただしn\geq 3)\]
が成り立つ。

$0 \leq I_{n+2} \leq I_{n+1} \leq I_{n}$から$I_{n+2}\leq I_{n}$である。$I_{n+2}\leq I_{n}$の両辺に$I_{n}$を足して$I_{n+2}+I_{n}\leq 2I_{n}$を得る。(1)より$I_{n}+I_{n+2}=1/(n+1)$であるから
\[\frac{1}{n+1} \leq 2 I_{n}\]
となる。この不等式の両辺に$n$をかけ,2で割ると
\[\frac{n}{2(n+1)} \leq I_{n}\]
を得る。
次に$I_{n} \leq I_{n-1} \leq I_{n-2}$から$I_{n}\leq I_{n-2}$である。$I_{n}\leq I_{n-2}$の両辺に$I_{n}$を足して$2I_{n}\leq I_{n-2}+I_{n}$を得る。(1)より$I_{n}+I_{n+2}=1/(n+1)$であるから
\[2 I_{n} \leq \frac{1}{n-1}\]
となる。この不等式の両辺に$n$をかけ,2で割ると
\[n I_{n} \leq \frac{n}{2(n-1)}\]
を得る。
以上より次の不等式が成り立つ。
\[\frac{n}{2(n+1)} \leq n I_{n} \leq \frac{n}{2(n-1)}.\]
さらに
\[\begin{aligned}
  \lim _{n \rightarrow \infty} \frac{n}{2(n+1)} &=\lim _{n \rightarrow \infty} \frac{1}{2\left(1+\frac{1}{n}\right)}=\frac{1}{2} \\
  \lim _{n \rightarrow \infty} \frac{n}{2(n-1)} &=\lim _{n \rightarrow \infty} \frac{1}{2\left(1-\frac{1}{n}\right)}=\frac{1}{2}
\end{aligned}\]
であるから,はさみうちの原理より
\[\lim _{n \rightarrow \infty} n I_{n}=\frac{1}{2}\]
である。
\\
\\
{\bf (4)の解答}

(1)より$I_{n}+I_{n+2}=1/(n+1)$であるから
\[I_{2k-1}+I_{2k+1}=\frac{1}{2k}\]
が成り立つ。両辺に$(-1)^{k-1}$をかけて
\[(-1)^{k-1}\left(I_{2 k-1}+I_{2 k+1}\right)=\frac{(-1)^{k-1}}{2 k}\]
を得る。これより$S_{n}$は次のようになる。
\[\begin{aligned}
  S_{n}=\sum_{k=1}^{n} \frac{(-1)^{k-1}}{2 k} &=\sum_{k=1}^{n}(-1)^{k-1}\left(I_{2 k-1}+I_{2 k+1}\right) \\
  &=\left(I_{1}+I_{3}\right)-\left(I_{3}+I_{5}\right)+\cdots+(-1)^{n-2}\left(I_{2 n-3}+I_{2 n-1}\right)+(-1)^{n-1}\left(I_{2 n-1}+I_{2 n+1}\right) \\
  &=I_{1}+(-1)^{n-1} I_{2 n+1}.
\end{aligned}\]
$I_{1}$の値は
\[\begin{aligned}
  I_{1} &=\int_{0}^{1} \frac{x}{x^{2}+1} d x \\
  &=\left[\frac{1}{2} \log \left(x^{2}+1\right)\right]_{0}^{1} \\
  &=\frac{1}{2} \log 2
\end{aligned}\]
である。これより$\displaystyle \lim _{n \rightarrow \infty} S_{n}$は
\[\lim _{n \rightarrow \infty} S_{n}=\frac{1}{2} \log 2+\lim _{n \rightarrow \infty}(-1)^{n-1} I_{2 n+1}\]
である。右辺の極限を求める。$0 \leq I_{n+1} \leq I_{n} \leq 1/(n+1)$より$0 \leq I_{n} \leq 1/(n+1)$である。$\displaystyle \lim _{n \rightarrow \infty} \frac{1}{n+1}=0$であるから,はさみうちの原理より$\displaystyle \lim _{n \rightarrow \infty} I_{n}=0$となる。よって
\[\lim _{n \rightarrow \infty}(-1)^{n-1} I_{2 n+1}=0\]
である。以上より
\[\lim _{n \rightarrow \infty} S_{n}=\frac{1}{2} \log 2\]
である。


\end{document}
