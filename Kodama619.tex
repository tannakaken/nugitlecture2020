\documentclass{jsarticle}
\usepackage{amsmath,amssymb}
\usepackage{cases}
\usepackage{enumerate}

\begin{document}
\begin{center}
	{\large 2020名古屋大学入試理系数学大問4}
\end{center}
\subsection*{問題}
2名が先攻と後攻にわかれ、次のようなゲームを行う。
\begin{enumerate}[(i)]
	\item 正方形の4つの頂点を反時計回りにA,B,C,Dとする。
				両者はコマを1つずつ持ち、ゲーム開始時には先攻の持ちゴマはA、後攻の持ちゴマはCにおいてあるとする。
	\item 先攻から始めて、交互にサイコロを振る。
				ただしサイコロは1から6までの目が等確率で出るものとする。
				出た目を3で割ったあまりが0のときコマは動かさない。
				また、余りが1のときは、自分のコマを反時計回りに隣りの頂点に動かす。
				もし移動した先に相手のコマがあれば、その時点でゲームは終了とし、サイコロを振った者の勝ちとする。
\end{enumerate}
ちょうど$n$回サイコロが振られたときに勝敗が決まる確率を$p_n$とする。
このとき、以下の問いに答えよ。
\begin{enumerate}[(1)]
	\item $p_2,p_3$を求めよ。
	\item $p_n$を求めよ。
	\item このゲームは後攻にとって有利であること、すなわち2以上の任意の整数$N$に対して、$\displaystyle \sum_{m=1}^{[\frac{N+1}{2}]}p_{2m-1}<\sum_{m=1}^{[\frac{N}{2}]}p_{2m}$が成り立つことを示せ。
				ただし正の実数$a$に対し$[a]$はその整数部分($k\leqq a<k+1$となる整数$k$)を表す。
\end{enumerate}

\subsection*{解答}

\end{document}